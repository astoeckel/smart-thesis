\documentclass[a4paper,11pt]{memoir}

\ifpdf
\usepackage[utf8]{inputenc}
\usepackage[T1]{fontenc}
\fi

\usepackage[english]{babel}

% This is the only file needed for actually using the smart-thesis style.
% Load the texgyrepagella font as main font and set the ``mathpazo'' font as
% math font
\usepackage{fontspec}
\usepackage{mathpazo}
\setmainfont
     [ BoldFont       = texgyrepagella-bold.otf ,
       ItalicFont     = texgyrepagella-italic.otf ,
       BoldItalicFont = texgyrepagella-bolditalic.otf,
       Numbers={OldStyle} ]
     {texgyrepagella-regular.otf}


\renewcommand{\baselinestretch}{1.5}

\usepackage[protrusion=true]{microtype}% Improves type setting. Requires a proper font.
\DeclareRobustCommand{\spacedallcaps}[1]{%
  \addfontfeature{LetterSpace=15,WordSpace=1.25}{\MakeTextUppercase{#1}}%
}%
\DeclareRobustCommand{\spacedlowsmallcaps}[1]{%
  \textsc{\addfontfeature{LetterSpace=15,WordSpace=1.25}{\MakeTextLowercase{#1}}}
}%

\makeatletter
\makechapterstyle{toledo}{%
  \chapterstyle{default}
  \renewcommand*{\chapterheadstart}{}
  \renewcommand*{\printchaptername}{%
    \centerline{\chapnumfont{\@chapapp\ \thechapter}}}
  \renewcommand*{\chapternamenum}{}
  \renewcommand*{\chapnumfont}{\normalfont\scshape\MakeTextLowercase}
  \renewcommand*{\printchapternum}{}
  \renewcommand*{\afterchapternum}{%
    \par\centerline{\parbox{0.5in}{\hrulefill}}\par}
  \renewcommand*{\printchapternonum}{%
    \vphantom{\chapnumfont \@chapapp 1}\par 
    \parbox{0.5in}{}\par}
  \renewcommand*{\chaptitlefont}{\normalfont\large}
  \renewcommand*{\printchaptertitle}[1]{%
    \centering \chaptitlefont\spacedallcaps{##1}}}

\nouppercaseheads
\makepagestyle{berlin}
\makeevenfoot{berlin}{\thepage}{}{}
\makeoddfoot{berlin}{}{}{\thepage}

\copypagestyle{chapter}{berlin}

%\if@twoside
    \makepsmarks{berlin}{%
      \def\chaptermark##1{%
        \markboth{\memUChead{%
          \ifnum \c@secnumdepth >\m@ne
            \if@mainmatter
              \@chapapp\ \thechapter. \ %
            \fi
          \fi
          ##1}}{}}%
      \def\tocmark{\markboth{\memUChead{\contentsname}}{\memUChead{\contentsname}}}%
      \def\lofmark{\markboth{\memUChead{\listfigurename}}{\memUChead{\listfigurename}}}%
      \def\lotmark{\markboth{\memUChead{\listtablename}}{\memUChead{\listtablename}}}%
      \def\bibmark{\markboth{\memUChead{\bibname}}{\memUChead{\bibname}}}%
      \def\indexmark{\markboth{\memUChead{\indexname}}{\memUChead{\indexname}}}%
      \def\sectionmark##1{%
        \markright{\memUChead{%
          \ifnum \c@secnumdepth > \z@
            \thesection. \ %
          \fi
          ##1}}}}
    \makepsmarks{berlin}{%
      \createmark{chapter}{left}{nonumber}{}{}
      \createmark{section}{right}{shownumber}{}{. \ }
      \createplainmark{toc}{both}{\contentsname}
      \createplainmark{lof}{both}{\listfigurename}
      \createplainmark{lot}{both}{\listtablename}
      \createplainmark{bib}{both}{\bibname}
      \createplainmark{index}{both}{\indexname}
      \createplainmark{glossary}{both}{\glossaryname}
    }
    \makeevenhead{berlin}{\spacedlowsmallcaps{\leftmark}}{}{}
    \makeoddhead{berlin}{}{}{\spacedlowsmallcaps{\rightmark}}
%\else
%    \makepsmarks{berlin}{%
%      \def\chaptermark##1{%
%        \markright{\memUChead{%
%          \ifnum \c@secnumdepth >\m@ne
%            \if@mainmatter
%              \@chapapp\ \thechapter. \ %
%            \fi
%          \fi
%          ##1}}}%
%      \def\tocmark{\markright{\memUChead{\contentsname}}}%
%      \def\lofmark{\markright{\memUChead{\listfigurename}}}%
%      \def\lotmark{\markright{\memUChead{\listtablename}}}%
%      \def\bibmark{\markright{\memUChead{\bibname}}}%
%      \def\indexmark{\markright{\memUChead{\indexname}}}}
%    \makepsmarks{berlin}{%
%      \createmark{chapter}{right}{shownumber}{\@chapapp\ }{. \ }
%      \createplainmark{toc}{right}{\contentsname}
%      \createplainmark{lof}{right}{\listfigurename}
%      \createplainmark{lot}{right}{\listtablename}
%      \createplainmark{bib}{right}{\bibname}
%      \createplainmark{index}{right}{\indexname}
%      \createplainmark{glossary}{right}{\glossaryname}
%    }
%    \makeoddhead{berlin}{\slshape\rightmark}{}{}
%\fi

\makeatother

\chapterstyle{toledo}
\pagestyle{berlin}

\strictpagecheck
\newcommand{\marginnote}[1]{\marginpar{
  \renewcommand{\baselinestretch}{1.25}
  \small
  \itshape
  \checkoddpage
  \ifoddpage
    \raggedright
  \else
    \raggedleft
  \fi
  {#1}
}}

\settypeblocksize{23.45cm}{11.85cm}{*}
%\setheaderspaces{1.29cm}{1cm}{*}
%\setheadfoot{1cm}{1cm}
\setulmargins{*}{*}{*}
\setmarginnotes{.6cm}{3.3cm}{1cm}
\checkandfixthelayout



% Contains some packages which are commonly used in a thesis.
% Note that these are optional.
\usepackage{csquotes} % Context sensitive quotation facilities. Recommended by babel and should be loaded before babel.

\usepackage[english]{babel} % Sets the language used. Essential for proper hyphenation. Translates key words like 'figure' or 'table'. Note that 'english' is American English.

\usepackage{latexsym,amsmath,amssymb,amsthm,amscd} % Provides math enviroments, math symbols and more things useful for math.

\usepackage{mathtools} % Enhances amsmath and provides further mathematical tools.

\usepackage{graphicx} % Provides inclusion of graphics and a proper interface for '\in­clude­graph­ics'.
% Su­per­sedes 'epsfig' and 'graphics'.

\usepackage{enumitem} % Provides control over list environments. Supersedes 'enumerate', which gives enumerate environment an optional argument which determines the style in which the counter is printed.

\let\newfloat\undefined % Hack to make floatrow work with memoir.
\usepackage{floatrow} % Handles alignment of floats (figures), with centering as default.

\usepackage{xspace} % Ugly hack that sometimes helps with otherwise missing spaces.

\usepackage[backgroundcolor=white,bordercolor=smartblue,textsize=tiny]{todonotes} % Todo notes.

% Use sans-serif font in todos to clearly distinguish it from other text
\makeatletter
\renewcommand{\todo}[2][]{\@bsphack\@todo[#1]{\sffamily{#2}}\@esphack\ignorespaces}
\makeatother

\usepackage[style=authoryear,backend=biber]{biblatex} % Bibliography.

\usepackage{algorithm} % Algorithms.

\usepackage{algpseudocode} % Algorithms.

\usepackage{varioref} % Automatically locates references on other pages. Load before hyperref.

\usepackage[hidelinks]{hyperref} % Clickable references.

\usepackage[all]{hypcap} % Anchors links to the beginning of their respective floats. Load after hyperref.

\usepackage[noabbrev,capitalize,nameinlink]{cleveref} % Names references automatically. Load after hyperref.

\usepackage[toc,acronym]{glossaries} % Glossary.

\usepackage{pgfplots}
\pgfplotsset{compat=1.8}
\pgfplotscreateplotcyclelist{smart}{%
  smartplum,semithick,every mark/.append style={fill=smartplum!80!black},mark=*\\%
  smartgreen,semithick,every mark/.append style={fill=smartgreen!80!black},mark=square*\\%
  smartorange,semithick,every mark/.append style={fill=smartorange!80!black},mark=otimes*\\%
  smartblue,semithick,mark=star\\%
  smartbutter,semithick,every mark/.append style={fill=smartbutter!80!black},mark=diamond*\\%
  smartred,semithick,every mark/.append style={solid,fill=smartred!80!black},mark=*\\%
}
\pgfplotsset{cycle list name=smart}



% Contains some macros which are commonly used in a thesis.
% Note that these are optional.
% Abbreviations
\newcommand*{\eg}{e.\,g.\@\xspace}
\newcommand*{\ie}{i.\,e.\@\xspace}
\newcommand*{\cf}{cf.\@\xspace}
\newcommand*{\etal}{et.\@ al.\@\xspace}
%\newcommand*{\etc}{etc.\@\xspace}
\makeatletter
\newcommand*{\etc}{%
  \@ifnextchar{.}%
    {etc}%
      {etc.\@\xspace}%
}
\makeatother

% Sets
\newcommand*{\R}{\mathbb{R}}
\newcommand*{\Q}{\mathbb{Q}}
\newcommand*{\Z}{\mathbb{Z}}
\newcommand*{\N}{\mathbb{N}}

% Matrix/vector operations
\newcommand*{\transpose}[1]{{#1^{\top}}}
\newcommand*{\inverse}[1]{{#1^{-1}}}
\newcommand*{\conjugate}[1]{{#1^{\ast}}}
\newcommand*{\pseudoinverse}[1]{{#1^{+}}}

% Stochastics
\newcommand*{\probability}[1]{{\mathrm{Pr}{#1}}}
\newcommand*{\expectation}[1]{{#1}}

% Optimization
\DeclareMathOperator*{\argmax}{arg\,max}
\DeclareMathOperator*{\argmin}{arg\,min}
\DeclareMathOperator{\lb}{lb}
\DeclareMathOperator{\sign}{sgn}

% These packages are only used in the demo and not necessarily
% required for smart-thesis.
\usepackage{blindtext}

% Loads bibliography for citations.
\addbibresource{demo-bibliography.bib}

% Starts creation of a glossary.
\makeglossaries
\newglossaryentry{mjollnir}{
  name=Mjölnir,
  description={the hammer of Thor}}

\newglossaryentry{wolf}{
  name=wolf,
  plural=wolves,
  description={an awesome animal and a subspecies of \emph{Canis lupus}}}
  


% Define all the metadata to be listed in \smarttitle and \smartcopyright
\thesistype{Master Thesis}
\discipline{Intelligent Systems}
\title{A Truly Magnificent Demo of the \LaTeX\ Smart Thesis Template With Respect to Long Titles and Incredibly Fascinating Blindtext}
\author{Jan Philip Göpfert,Andreas Stöckel}
\institution{University of GNU,Faculty of Templates,\TeX\ Group}
\supervisors{Prof.\@~Dr.-Ing.\@~Jane Doe,M.Sc.\@~John Doe}
\internalid{M13}

\begin{document}

\frontmatter

\smarttitle

\smartcopyright{This LaTeX template is published under the Creative Commons Zero license. To the extent possible under law, the authors have waived all copyright and
related neighboring rights to Smart Thesis. This work is published from: Germany.}

\smartinitialepigraph{The secret of getting ahead is getting started.}{Mark Twain}

\tableofcontents

\mainmatter

\chapter{A multitude of tests}

\epigraph{Science is what we understand well enough to explain to a computer. Art is everything else we do.}{Donald E. Knuth}

\blindtext

\begin{figure}
    \small
    \centering
    \begin{tikzpicture}
       \begin{axis}
          \addplot coordinates{(0,0) (1,2)};
          \addplot coordinates{(0,1) (1,3)};
          \addplot coordinates{(0,2) (1,4)};
          \addplot coordinates{(0,3) (1,5)};
          \addplot coordinates{(0,4) (1,6)};
          \addplot coordinates{(0,5) (1,7)};
       \end{axis}
    \end{tikzpicture}
    \caption{Just a fancy test for auto-coloring of graphs}
\end{figure}

\section{Some math}

\marginnote{This formula is also known as \emph{Pythagorean Theorem}, albeit it is usually written with the variables $a$, $b$, and $c$.}
Here follows some tests for typesetting math. Note how we use the \texttt{\\marginnote} command to add some pretty notes which appear in the margin. This is quite nice, as it allows a more narrow width of the text body without the page looking too sparse\footnote{This is a footnote}. Here comes our first equation
\begin{align}
  z^2 &= x^2 + y^2 \\
  z &= \sqrt{x^2 + y^2} \,.
\end{align}

\pagebreak
\marginnote{Did you know? The discrete fourier transformation shown in equation \ref{eqn:dft} is the backbone of the modern information-society.}
And here comes some more math, this time a discrete Fourier transformation (DFT)  which is a little bit more sophisticated than the last example\footnote{More is always better!}:
\begin{align}
	f_m =  \sum_{k=0}^{2n-1} x_k \, e^{-\frac{2\pi i}{2n} mk } \label{eqn:dft}
\qquad
m = 0,\dotsc,2n-1 \,.
\end{align}
For words of science, see \cite{botsch2010polygon}. Unfortunately, that book has nothing about \glspl{wolf}.

\begin{figure}
	\small
	\centering
	\begin{tikzpicture}
		\begin{axis}[
			axis lines=left,
			scaled ticks=false,
			xticklabel style={
				rotate=90,
				anchor=east,
				/pgf/number format/precision=3,
				/pgf/number format/fixed,
				/pgf/number format/fixed zerofill,
			},
		]
		% density of Normal distribution:
		\newcommand\MU{0}
		\newcommand\SIGMA{0.1}
		\addplot+[mark=none,domain=-3*\SIGMA:3*\SIGMA,samples=201]
		{exp(-(x-\MU)^2 / 2 / \SIGMA^2)
			/ (\SIGMA * sqrt(2*pi))};
		\end{axis}
	\end{tikzpicture}
	\caption{Gaussian distribution. This plot shows a guassian distribution with mean $\mu = 0$ and standard deviation $\sigma = 0.1$. As clearly visible, the value of the distribution never reaches zero.}
	\label{fig:gaussian_distr}
\end{figure}


\section{Even more tests}

\marginnote{Blood-red were his spurs i' the golden noon; wine-red was his velvet coat, when they shot him down on the highway, down like a dog on the highway, and he lay in his blood on the highway, with the bunch of lace at his throat.}
\blindtext

\subsection{A test of three parts}

\blindtext

\chapter{Another test chapter}

\begin{algorithm}
	\small
	\begin{shaded}
		\begin{algorithmic}[1]
			\newcommand*{\To}{\textbf{to}\xspace}
			\newcommand*{\Init}{\State\textbf{init}\xspace}
			\newcommand*{\B}{\mathcal{B}}
			\Init \(M \gets 0 \in \B^{N \times r}\) \Comment{Result matrix}
			\For{ \(i \gets 1\) \To N}
				\For{ \(j \gets r - r_1 + 1\) \To \(r\)}
					\State \(\ell \gets \) \Call{RandomSelect}{$\{1, \ldots, j - 1\}$} \Comment{Uniformly select set entry}
					\If{\({M}[i,\ell] = 1\)}
						\State \({M}[i,j] \gets 1\)
					\Else
						\State \({M}[i,\ell] \gets 1\)
					\EndIf
				\EndFor
			\EndFor
		\end{algorithmic}
	\end{shaded}
	\caption[Uncorrelated random data generation]{Algorithm for the generation of a block $M$ of $N$ uncorrelated random vectors of length $r$, containing exactly $r_1$ ones.}
	\label{alg:binam_random_data}
\end{algorithm}

\epigraph{To test is to live. To believe is to die.}{Anonymous Heretic}
\marginnote{Yes, this is just a test. If you read this, you are most likely doomed.}
\blindtext

\begin{figure}
    \centering
    \includegraphics[width=8cm]{demo}
    \caption{Truly a fine canister, comparable only to \gls{mjollnir}.}
    \label{fig:canister}
\end{figure}

\Blindtext\Blindtext

A picture says more than a thousand words (see \ref{fig:canister}).

\section{The Throat of the World}

\marginnote{Margin notes help to add structure to long runs of text}
\blindtext

\blindlistlist[2]{itemize}

\section{The Silence Has Been Broken}

\marginnote{Test test test}\blindtext

\blindlistlist[2]{enumerate}

\section{Discerning the Transmundane}

\marginnote{You can even add small formulas, like this one:
\begin{align*}
\sum_{i=1}^{n} i &= n \cdot \frac{n + 1}2
\end{align*}}
\blindtext

\subsection{Composure, Speed, and Precision}

\marginnote{Or this one:
\begin{align*}
E &= mc^2
\end{align*}}
\blindtext

\backmatter
\printglossaries
\printbibliography[heading=bibintoc]

\end{document}

